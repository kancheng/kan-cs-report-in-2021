\chapter{总结}

此报告根据调查与程序代码两方向做各自的呈现,此作业尽量力求兼顾两者,第一章说明作业目标、第二章撰写论文心得与论文全文翻译与精读、第三章处理哈佛大学自然语言处理实验室的文件心得含程序代码改版测试、全文翻译精读,并简述说明过程使用 LaTeX 与 Docker 来改善自身目前的研究工作流程跟说明为了完成作业所阅读的众多技术文件与网络资源 ,而第四章则分为三大部分,第一部分为 Transformer 理解、第二部分为论文文献综述还有调研准备工作包含阅读的研究,第三部分则为自己所选择 复现地论文。最后才是此章回馈与总结。

\section{课程回馈}

\begin{Verbatim}
Feedback(5pt)
Time you spend for this assignment, i.e., how many hours? (1pt)
Comments for this course? (1pt)
Comments for this assignment? (1pt)
Suggestion for the lectures of next year? (2pt)
\end{Verbatim}

1. Time you spend for this assignment, i.e., how many hours?

第二章的论文与第三章的技术文件所涉略的技术文件与资源收集,从 9 月始开学作业标准公布后就慢慢进行搜集,在 10 月时抽出各周周末慢慢查技术文件等资源去了解 Transformer 的运作机制、原理、主流前沿研究的发展,在 10 月时就已经完成了第二章论文精读的草稿并做出笔记,而到了 11 月 开始查 Transformer 的程式码,并开始尝试哈佛实验室技术文件的问题,最后于 11 底完成第一版本的作业。同时本作业的 LaTeX 北京大学模板与 Docker 使用,甚至其他技术部分则是根据科技论文写作、计算机视觉、机器学习、数位媒体技术等不同课程作业探索而来的集大成,这些成果也是于这学期不同的周末所完成。最后则是第四章的三大部分,其进度于 12 月开工,但由于身体健康不佳的缘故,导致必须作息回归正常,整体的进度才缓下来,最后只能把握不同的时间进行准备。

总而言之,该作业横跨整个学期不同时段跟周末,具体花了多少小时很难进行计算,但在完成这份作业的路上学生写得很尽兴,而且也利用这个接触了一些自己平常不会去了解的有趣技术,若时间够长学生还可以写的更完善、更完美,当然这份作业会因为研究的需求,后续工作会继续进行下去。


2. Comments for this course?

我其实写得很开心,但如果可以,我会希望能有课堂录影之类的手段,让有机会重复复习,因为自己的基础跟能力有限,往往都是课后慢慢阅读,甚至要读好一段时间才能理解。很多时候看到同在教室的同济们的优秀表现,让学生羡慕不已。

另一个希望推广组成团队,全班同学一起进行共笔等,所谓的共笔是台湾医学院学生们的文化,一群人一起集思广益变强,学生相信一群人一起写协作写笔记爬文讨论的感觉必然不一样。


3. Comments for this assignment? 

希望下次作业细节条件能够更早确定。作业条件的变化会影响学生排时间的时间表,而时间表的改动则会影响学生的时间安排紧凑程度,从而导致作息状况。


4. Suggestion for the lectures of next year?

下次可以建议课程模组化,分为必须掌握的基础跟老师自由发挥部分,基础部分预先录制好,这样老师即便面对跟不上学生也没关系,当然这必须要有规划自由发挥的部分,这样学生就必须来上课,确保出席率。若配合建议的共笔或许会更有成效。


