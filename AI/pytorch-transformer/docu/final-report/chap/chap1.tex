\chapter{作业目标}
\label{chap:1}

\section{作业说明}

该作业为人工智慧期末报告,其专案为 kancheng/kan-cs-report-in-2021 ,程式码则可于 kan-cs-report-in-2021/AI/pytorch-transformer/code 中查询,而文件则可由专案中进行查询。

\section{论文作业目标}

Step 1 : Read the paperվAttention is all you needտand write a reading report. 

Step 2: Read and run Transformer-related code (NLP Harvard).

Recording the operation result and your understanding in the experimental report.

Step 3: You can choose a specific area (No restriction on direction), and apply the transformer to this area.

It is necessary to find a published paper and successfully reproduce the corresponding result in the paper.Recording the related paper, experimental steps, and your results vs results in the paper to the experimental report.

Step 4: 

The reading report \& source code \& the related paper \& experimental report will be organized and submitted. 


\section{The Annotated Transformer}


Q1: NO GPU available

\begin{itemize}
\item A1: a survey about transformer
\item Example: https://arxiv.org/pdf/1809.02165.pdf
\end{itemize}


Q2: code plagiarism

\begin{itemize}
\item In particular, all code and documentation should be entirely your own work. You may consult with other students about high-level design strategies related to programming assignments, but you may not copy code or use the structure or organization of another student's program.
\item If you use any code or functions found from the internet, please tell us the reference link and how do you use it. Direct code copy from the internet would be considered violation of this policy.
\item If we find there are two returned assignments same in large proportional code, both of the assignments would be considered violation of this policy.
\end{itemize}

